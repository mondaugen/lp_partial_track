% --------------------------------------------------------------------------
% Template for WASPAA-2017 paper; to be used with:
%          waspaa17.sty  - WASPAA 2017 LaTeX style file, and
%          IEEEbib.bst - IEEE bibliography style file.
%
% --------------------------------------------------------------------------

\documentclass{article}
\usepackage{waspaa17,amsmath,graphicx,url,times}
%\usepackage{waspaa17,amssymb,amsmath,graphicx,times,url}
\usepackage{color}
\usepackage[ruled]{algorithm2e}
\usepackage{amsmath}
\usepackage{amsfonts}
\usepackage{amssymb}
\usepackage[labelsep=period]{caption}
%\usepackage{mathabx}

% Example definitions.
% --------------------
\def\defeqn{\stackrel{\triangle}{=}}
\newcommand{\symvec}[1]{{\mbox{\boldmath $#1$}}}
\newcommand{\symmat}[1]{{\mbox{\boldmath $#1$}}}
\newcommand\ddfrac[2]{\frac{\displaystyle #1}{\displaystyle #2}}
%\DeclareMathOperator{\argmin}{argmin}
%\DeclareMathOperator{\argmax}{argmax}
\def\argmin{\mathop{\rm arg\,min}}
\def\argmax{\mathop{\rm arg\,min}}
\newcommand\ImNe[1]{\Im \left\{ #1 \right\}}
\newcommand\ReNe[1]{\Re \left\{ #1 \right\}}
\newcommand{\D}{\displaystyle}
\newcommand{\T}{\textstyle}
\newcommand{\TmpCapTitle}[0]{default}
\DeclareCaptionFormat{inc-cap-title}{#1: \textbf{\TmpCapTitle#2}#3\par}
\DeclareCaptionFormat{no-cap-title}{#1\textbf{#2}#3\par}
\newcommand{\CaptionWithTitle}[2]{
    \renewcommand{\TmpCapTitle}[0]{\protect#1}
    \captionsetup{format=inc-cap-title}
    \caption[\protect#1.]{#2}
%    \caption{#2}
}
\captionsetup{format=no-cap-title}
\newcommand{\MyInput}[1]{\immediate\write18{cat #1}}
\def\figwidthscale{0.5}

% symbol definitions
\newcommand{\cLP}{\boldsymbol{c}_{\rho}}
\newcommand{\BS}[1]{\boldsymbol{#1}}

% Title.
% --------------------
\title{A linear programming approach to the tracking of partials}

%% Single addresses (uncomment and modify for single-address case).
%% --------------------
%\name{Author(s) Name(s)\thanks{Thanks to XYZ agency for funding.}}
%\address{Author Affiliation(s)}
%%
%% For example:
%% ------------
%%\address{School\\
%%       Department\\
%%       Address}

% Two addresses
% --------------------
\twoauthors
  {Nicholas Esterer}%\sthanks{}}
    {Schulich School of Music, McGill University, \\
     Music Technology Area \& CIRMMT \\
     555 Sherbrooke Street West \\
     Montreal, Quebec \\
     Canada H3A 1E3 \\
     nicholas.esterer@mail.mcgill.ca}
 {Philippe Depalle}%\sthanks{}}
    {Schulich School of Music, McGill University, \\
     Music Technology Area \& CIRMMT \\
     555 Sherbrooke Street West \\
     Montreal, Quebec \\
     Canada H3A 1E3 \\
     philippe.depalle@music.mcgill.ca}

\begin{document}

\ninept
\maketitle

\begin{sloppy}

\begin{abstract}
    A new approach to the tracking of sinusoidal chirps using linear programming
    is proposed. It is demonstrated that the classical algorithm of
    \cite{mcaulay1986speech} is greedy and exhibits exponential complexity for
    long searches. A linear programming (LP) formulation to find the best $L$ paths
    in a lattice is described and its complexity is analysed. Finally it is
    demonstrated that the new LP formulation outperforms the classical algorithm
    in the tracking of sinusoidal chirps in high levels of noise --- both in
    computational complexity and robustness.
\end{abstract}

\begin{keywords}
partial tracking, linear programming, optimization, additive synthesis
\end{keywords}

\section{Partial Tracking\label{chap:partialtracking}}

%In the previous paper, we saw how to estimate parameters of sinusoids with
%polynomial phase. While theoretically applicable to signals of arbitrary length,
%for reasons of flexibility and efficiency, we usually estimate the local
%parameters of the signal under a low-order model and connect multiple
%estimations to form a partial. We will call these local estimations ``analysis
%points'' or ``parameter sets''.

This paper presents an interpretation of the \textit{peak matching} procedure
of McAulay and Quatieri \cite{mcaulay1986speech}, a classical approach to discovering partials. Our
interpretation allows for the specification of an arbitrary cost function
measuring the plausibility that a set of analysis points forms the path of a
partial. With this path interpretation, we were able to design a technique that finds the
optimal set of paths under a constraint on the number of paths. The paper 
concludes with an example of partial tracking on a synthetic signal.

Typically the discrete time short-time Fourier transform (DTSTFT) is computed
for a block of contiguous samples, called a \textit{frame} and these frames are
computed every $H$ samples, $H$ being the \textit{hop-size}. We will denote the
$N_{k}$ sets of parameters at local maxima in frame $k$ as $\theta_0^{k},
\dotsc, \theta_{N_{k}-1}^{k}$ and the $N_{k+1}$ in frame $k+1$ as
$\theta_0^{k+1}, \dotsc, \theta_{N_{k+1}-1}^{k+1}$ where $k$ and $k+1$ refer to
adjacent frames. We are interested in paths that extend across $K$ frames where
each path touches only one parameter set and each parameter set is either
exclusive to a single path or is not on a path.

\subsection{Note on notation}

In this paper, indexing starts at 0. If we have a vector $\BS{x}$ then
$\BS{x}_{i}$ is the $i$th row or column of that vector depending on the
orientation. The same notation is used for Cartesian products, e.g., if $\alpha$
and $\beta$ are sets and $A =
\alpha \times \beta$ then for the pair $a \in A$ $a_{0}$ is the first item in
the pair and $a_{1}$ the second.

The number of frames is denoted $K$, the number of nodes in the $k$th frame $N_k$ and
the total number of nodes $M = \sum_{k=0}^{K-1} N_{k}$. $\theta_{i}^{j}$ is the
$i$th node of the $j$th frame, $\theta^{j}$ the set of all the nodes in the
$j$th frame and $\theta_{m}$ the $m$th node out of all $M$ nodes ($0 \leq m <
M$).

%In Section~\ref{sec:mqfmfromphase} we discussed how to determine reasonable
%values for the coefficients of a cubic phase polynomial by using the frequency,
%phase and time difference of two local maxima in the DTSTFT. In this section we
%discuss possible ways of determining which local maxima are connected. This is
%referred to as \textit{peak matching} \cite{mcaulay1986speech}
%or \textit{partial tracking} \cite{smith1987parshl} \cite{depalle1993tracking}.

\section{A greedy method}

In this section, we present the McAulay-Quatieri method of peak matching. It is
conceptually simple and a set of short paths can be computed quickly, but
it can be sensitive to spurious peaks and is optimal only in the sense that
the set of paths computed contains the best path possible --- the quality of the
other paths may be compromised under this criterion.

In \cite[q.~748]{mcaulay1986speech} the peak matching algorithm is described in
a number of steps; we summarize them here in a way comparable with the linear
programming formulation to be presented shortly. In that paper, the parameters
of each data point are the instantaneous amplitude, phase, and frequency but we
will allow for arbitrary parameter sets $\theta$.  Define a distance function
$\mathcal{D}
\left( \theta_{i},\theta_{j} \right)$ that computes the
similarity between $2$ sets of parameters. We will now consider a method that
finds $L$ tuples of parameters that are closest.

We compute the cost tensor
$
    \BS{C} = \theta^{k}
    \otimes_{\mathcal{D}} \mathellipsis \otimes_{\mathcal{D}} \theta^{k+K-1}
$.
For each $l \in \left[0 \dotsc L-1 \right]$, find the indices
$i_{0},\dotsc,i_{K-1}$ 
corresponding to the shortest
distance, then remove the $i_{0},\dotsc,i_{K-1}$th rows (lines of table entries)
in the their respective dimensions from consideration
and continue until $L$ tuples have been determined or the distances exceed some
threshold $\Delta_{\text{MQ}}$. This is summarized in Algorithms~\ref{alg:mq_peak_match}

\begin{algorithm}
    \KwIn{the cost matrix $\BS{C}$}
    \KwOut{$L$ tuples of indices $\Gamma$, or fewer if $\Delta_{\text{MQ}}$ exceeded}
    $\Gamma \leftarrow \varnothing$\;
    \For{$l \leftarrow 0$ to $L-1$}{
        $\D \Gamma_{l}=\argmin_{[0,\dotsc,M_{0}-1] \times
        \mathellipsis \times [0,\dotsc,M_{K-1}-1] \setminus \Gamma}
            \BS{C}$\;
            \If{$ C_{\Gamma_{l}} > \Delta_{\text{MQ}}$}{
            \KwRet{$\Gamma$}
        }
        $\Gamma \leftarrow \Gamma \cup C_{\Gamma_{l}}$\;
    }
    \KwRet{$\Gamma$}
    \caption{A generalized McAulay-Quatieri peak-matching algorithm.}%
    \label{alg:mq_peak_match}
\end{algorithm}

This is a greedy algorithm because on every iteration the smallest cost is
identified and its indices are removed from consideration. Perhaps choosing a
slightly higher cost in one iteration would allow smaller costs to be chosen in
successive iterations. This algorithm does not allow for that. In other terms,
the algorithm does not find a set of pairs that represent a globally minimal sum of
costs.
Furthermore, the algorithm does not scale well: assuming equal numbers of
parameter sets in all frames, the search space grows exponentially with
$K$. Nevertheless, the method is simple to implement, computationally negligible
when $K$ is small, and works well with audio signals \cite{mcaulay1986speech}
\cite{smith1987parshl}.

\section{$L$ best paths through a lattice via linear programming (LP)}

In this section we show how to find L paths through a lattice of $K$ frames such
that the sets of nodes on each path are disjoint. The $k$th frame of the lattice
contains $N_{k}$ nodes for a total of $M = \sum_{k=0}^{K-1}N_{k}$ nodes.

Whereas the limiting cost under which a path would be considered was the sum of
the costs of all connections in the McAulay-Quatieri method, for the LP method
we define the cost $\Delta_{\text{LP}}$ as the limiting cost under which the
connection between two nodes will be considered.

The solution vector $\BS{x}$ to the linear program shall indicate the presence of a
connection between a pair of nodes by having an entry equal to $1$ and otherwise
have entries equal to $0$. To enumerate the set of possible connection-pairs we
define
\begin{equation}
    \rho = \left\{ (i,j) : \mathcal{D}(\theta_{i},\theta_{j}) \leq
    \Delta_{\text{LP}} , 0 \leq i < M, 0 \leq j < M, i \neq j \right\}
\end{equation}
The cost vector of the objective function is then 
\begin{equation}
    \cLP = \left\{ D(\theta_{i},\theta_{j}) \forall (i,j) \in
    \rho \right\}
\end{equation}
and the length of $\cLP$ is $\# \rho = \# \cLP = P$, in other words, $P$
pairs of nodes. For convenience we define a bijective mapping ${\mathcal{B} :
\rho \rightarrow [0, \mathellipsis, M-1]}$ giving the index in $\BS{x}$ of the
pair $p \in \rho$. For the implementation considered in this paper,
$\mathcal{D}(\theta_{i},\theta_{j}) = \infty$ for all $i,j$ not in adjacent
frames and so $P$ will be no larger than $(K-1)N^{2}$ (assuming the same
number of nodes $N$ in each frame).

The total cost of the paths in the solution is then calculated through the
inner product $\cLP^{T}\BS{x}$. To obtain $\BS{x}^{\ast}$ that represents $L$
disjoint paths we must place constraints on the structure of the solution. Some
of the constraints presented in the following are redundant but the redundancies
are kept for clarity; later we will show which constraints can be removed
without changing the optimal solution $\BS{x}^{\ast}$.

All nodes in $\BS{x}^{\ast}$ will have at most one incoming connection or
otherwise no connections, a constraint that can be enforced through the
following linear inequality. Define $\BS{A}^{\text{I}} \in
\mathbb{R}^{R_{\text{I}} \times P}$ with $R_{\text{I}} = \sum_{k=1}^{K-1}
N_{k}$, the number of nodes in all the frames excluding the first. We sum all
the connections into the node $r_{\text{I}} + N_{0}$ represented by the
respective entry in $\BS{x}$ through an inner product with the $r_{\text{I}}$th
row in $\BS{A}^{\text{I}}$ and require that this sum be between $0$ and $1$,
i.e.,
\begin{equation}
    \BS{A}^{\text{I}}_{r_{\text{I}},\mathcal{B}(p)} = \begin{cases}
        1 & \text{if } p_{1} = r_{\text{I}}+N_{0} \\
        0 & \text{otherwise}
    \end{cases}, 0 \leq r_{\text{I}} < R_{\text{I}}, p \in \rho
\end{equation}
and
\begin{equation}
    \BS{0} \leq \BS{A}^{\text{I}}\BS{x} \leq \BS{1}
\end{equation}
Similarly, to constrain the number of outgoing connections into each node, we
define $R_{\text{O}} = \sum_{k=0}^{K-2} N_{k}$ and
$\BS{A}^{\text{O}} \in \mathbb{R}^{R_{\text{O}} \times P}$ with
\begin{equation}
    \BS{A}^{\text{O}}_{r_{\text{O}},\mathcal{B}(p)} = \begin{cases}
        1 & \text{if } p_{0} = r_{\text{O}} \\
        0 & \text{otherwise}
    \end{cases}, 0 \leq r_{\text{O}} < R_{\text{O}}, p \in \rho
\end{equation}
and
\begin{equation}
    \BS{0} \leq \BS{A}^{\text{O}}\BS{x} \leq \BS{1}
\end{equation}

To ensure no breaks in the paths it is required that the number of incoming
connections into a given node equal the number of outgoing connections for the
$R_{\text{B}} = \sum_{k=1}^{K-2} N_{k}$ nodes potentially having both incoming
and outgoing connections.
\begin{equation}
    \BS{A}^{\text{B}}_{r_{\text{B}}} = \BS{A}^{\text{B}}_{r_{\text{B}}} -
    \BS{A}^{\text{B}}_{r_{\text{B}}+N_{0}} \text{ for rows } 0 \leq r_{\text{B}}
    < R_{\text{B}}
\end{equation}
and
\begin{equation}
    \label{eq:cxnbalcon}
    \BS{A}^{\text{B}}\BS{x} = \BS{0}
\end{equation}

Finally we ensure that there are $L$ paths by counting the number of connections
in each frame and constraining this sum to be $L$. We choose arbitrarily to
count the number of outgoing connections by summing rows of $\BS{A}^{\text{O}}$
into rows of $\BS{A}^{\text{C}} \in \mathbb{R}^{(K-1) \times P}$
\begin{equation}
    \BS{A}^{\text{C}}_{r_{\text{C}}} = \sum_{k=a}^{b} \BS{A}^{\text{O}}_{k}
\end{equation}
with $a = \sum_{j=0}^{r_{\text{C}}} N_{j}$ and $b = \sum_{j=0}^{r_{\text{C}}+1}
N_{j}$ and
\begin{equation}
    \label{eq:cxnlcon}
    \BS{A}^{\text{C}}\BS{x} = L\BS{1}
\end{equation}

As stated above, some of these constraints are redundant and can be removed.
Indeed, we have $\BS{0} \leq \BS{x} \leq \BS{1}$, therefore we will always have
$\BS{A}^{\text{I}}\BS{x} \geq 0$ and $\BS{A}^{\text{O}}\BS{x} \geq 0$.
Furthermore, all but the last row of (\ref{eq:cxnlcon}) can be seen as
constructed from linear combinations of rows of (\ref{eq:cxnbalcon}) and the last
row of (\ref{eq:cxnlcon}) so we only require $\BS{A}^{\text{C}}_{K-2}\BS{x} = L$.
Finally we always have $\BS{x} \leq \BS{1}$ because of the constraint that there
be a maximum of $1$ incoming and outgoing connection from each node.

The complete LP to find the $L$ best disjoint paths through a lattice described
by node connections $\rho$ is then
\[
    \min_{\BS{x}} \cLP^{T} \BS{x} 
\]
subject to
\[
    \BS{G}\BS{x} =
    \begin{bmatrix}
        \BS{A}^{\text{I}} \\
        \BS{A}^{\text{O}} \\
        -\BS{I}
    \end{bmatrix} \BS{x} \leq
    \begin{bmatrix}
        \BS{1} \\
        \BS{1} \\
        \BS{0}
    \end{bmatrix}
\]
\begin{equation}
    \label{eq:lpprogfull}
    \BS{A}\BS{x} =
    \begin{bmatrix}
        \BS{A}^{\text{B}} \\
        \BS{A}^{\text{C}}_{K-2}
    \end{bmatrix} \BS{x} = 
    \begin{bmatrix}
        \BS{0} \\
        L
    \end{bmatrix}
\end{equation}
where $\BS{I}$ is the identity matrix.

\section{Memory complexity}

To simplify notation, in this section we assume there are $N$ nodes in each
frame of the lattice.

Although the matrices involved in (\ref{eq:lpprogfull}) are large, only a small
fraction of their values are non-zero. As stated above, the number of variables in $\BS{x}$ is
$P = (K-1)N^{2}$. Even though matrices $\BS{A}^{\text{I}}, \BS{A}^{\text{O}} \in
\mathbb{R}^{N(K-1) \times P}$, each contains only $P$ non-zero
entries.  $\BS{A}^{\text{B}} \in \mathbb{R}^{N(K-1) \times P}$ but contains only
$2N^{2}(K-2)$ non-zero entries while $\BS{A}^{\text{C}}_{K-2} \in \mathbb{R}^{P}$
contains merely $N$. The $\BS{x} \geq \BS{0}$ constraint requires a matrix
with $P$ non-zero entries. The total memory complexity is therefore $4P +
2N^{2}(K-2)$ floating-point numbers.

\section{Complexity}

Here we will compare the complexity of the LP formulation of the best $L$ paths
search to the greedy McAulay-Quatieri method as well a combinatorial algorithm
proposed in \cite{wolf1989finding}.

Assuming the same number of nodes $N$ in each frame of the lattice, the 
search for the $l$th best path in the generalized McAulay-Quatieri
algorithm ($0 \leq l < L$) requires a search over $(N-l)^{K}$ possible paths.

The LP formulation of the $L$-best paths problem gives results equivalent to the
solution to the $L$-best paths problem proposed in \cite{wolf1989finding}. The
complexity of the algorithm by Wolf in \cite{wolf1989finding} is equivalent to
the Viterbi algorithm for finding the single best path through a trellis whose
$k$th frame has $\binom{N_{k}}{L}\binom{N_{k+1}}{L}L!$ connections where $N_{k}$
and $N_{k+1}$ are the number of nodes in two consecutive frames of the original
lattice. Therefore, assuming a constant number $N$ of nodes in each frame, its
complexity is $O((\binom{N}{L}^{2}L!)^{2}K)$.

The complexity of the algorithm presented here is polynomial in the number of
variables (the size of $\BS{x}$).  Assuming we use the algorithm in
\cite{karmarkar1984new} to solve the LP, our program has a complexity of
$O(P^{3.5}B^{2})$ where $B$ is the number of bits used to represent each number
in the input.  However, this bound is conservative considering the reported
complexity of modern algorithms.

For instance, the complexity of a log-barrier interior-point method is dominated
by solving the system of equations
\begin{equation}
    \label{eq:kkt}
    \begin{bmatrix}
        -\BS{D}\BS{G}^{T}\BS{G} & \BS{A}^{T} \\
        \BS{A} & \BS{0}
    \end{bmatrix}
    \begin{bmatrix}
        \BS{u} \\
        \BS{v}
    \end{bmatrix}
    =
    \begin{bmatrix}
        t\cLP + \BS{A}^{T}\BS{d} \\
        \BS{0}
    \end{bmatrix}
\end{equation}
some $10$s of times \cite[p.~590]{boyd2004convex}. 
Each iteration then takes ${\frac{2}{3}((K-1)N^{2} + (K-2)N)^{3}}$ flops (floating-point
operations) to solve ($\ref{eq:kkt}$) using a standard $LU$-decomposition
\cite[p.~98]{golub1996matrix}. As
$\BS{D}$ is a diagonal matrix, if
the number of nodes in each frame is $N$ for all frames, then
$\BS{D}\BS{G}^{T}\BS{G}$ will be a block-diagonal matrix made up of $K-1$ blocks
$\BS{B}_k \in \mathbb{R}^{N^{2}\times N^{2}}$. The system can then be solved in
\begin{multline*}
    \frac{2}{3}(K-1)N^{6} + 2(K-2)(K-1)N^{5} + \\
    2(K-2)^{2}(K-1)N^{4} + \frac{2}{3}(K-2)^{3}N^{3}
\end{multline*}
flops \cite[p.~675]{boyd2004convex}; this complexity is without exploiting the
sparsity of $\BS{A}$ nor the structure of $\BS{B}_k = \BS{D}_{k}\BS{C}$ --- the product
of some diagonal matrix $\BS{D}_{k}$ with an unchanging symmetric matrix
$\BS{C}$.

\section{Partial paths on an example signal\label{sec:mq_lp_compare_chirp}}

We compare the greedy and LP based methods for peak matching on a synthetic
signal. The signal is composed of $Q=6$ chirps of constant amplitude, the $q$th
chirp $s$ at sample $n$ described by the equation
\[
    s_{q}(n) = \exp(j(\phi_{q} + \omega_{q}n +
    \frac{1}{2} \psi_{q} n^{2}))
\]
The parameters for the 6 chirps are presented in
Table~\ref{tab:ptrackexamplechirpparams}.

\begin{table}[!b]
    \caption{Parameters of $q$th chirp. $\nu_{0}$ and $\nu_{1}$ are the initial and
    final frequency of the chirp in Hz. \label{tab:ptrackexamplechirpparams}}
    \begin{center}
        \begin{tabular}{l c c c c c}
            $q$ & $\phi_{q}$ & $\omega_{q}$ & $\psi_{q}$ & $\nu_{0}$ & $\nu_{1}$ \\
            \hline
            \input{plots/mq_lp_compare_chirp_params.txt}
        \end{tabular}
    \end{center}
\end{table}

Two 1 second long signals are synthesized at a sampling rate of 16000 Hz, the
first with chirps 0--2, the second with chirps 3--5. We add
Gaussian distributed white noise at several SNRs to evaluate the technique in the
presence of noise.

%\begin{figure}[!t]
%    %\centering
%    \centering
%    \includegraphics[width=\figwidthscale\textwidth]{plots/mq_lp_compare_chirp_20.eps}
%    \CaptionWithTitle{%
%        \input{plots/mq_lp_compare_chirp_20.txt}%
%    }{ Line-segments representing the frequency and frequency-slope at local
%        spectrogram maxima. In the bottom two plots the line segments not deemed
%        by the respective algorithms as belonging to a partial path are
%        discarded, revealing the estimated partial trajectories. See
%        Table~\ref{tab:ptrackexamplechirpparams} for the chirp parameters.
%    \label{plot:mq_lp_compare_chirp_20}}
%\end{figure}
%\begin{figure}[!t]
%    %\centering
%    \centering
%    \includegraphics[width=\figwidthscale\textwidth]{plots/mq_lp_compare_chirp_15.eps}
%    \CaptionWithTitle{%
%        \input{plots/mq_lp_compare_chirp_15.txt}%
%    }{ Line-segments representing the frequency and frequency-slope at local
%        spectrogram maxima. In the bottom two plots the line segments not deemed
%        by the respective algorithms as belonging to a partial path are
%        discarded, revealing the estimated partial trajectories. See
%        Table~\ref{tab:ptrackexamplechirpparams} for the chirp parameters.
%    \label{plot:mq_lp_compare_chirp_15}}
%\end{figure}
\begin{figure}[!t]
    %\centering
    \centering
    \centerline{\includegraphics[width=\figwidthscale\textwidth]{plots/mq_lp_compare_chirp_10.eps}}
    \CaptionWithTitle{%
        \input{plots/mq_lp_compare_chirp_10.txt}%
    }{ Line-segments representing the frequency and frequency-slope at local
        spectrogram maxima. In the bottom two plots the line segments not deemed
        by the respective algorithms as belonging to a partial path are
        discarded, revealing the estimated partial trajectories. See
        Table~\ref{tab:ptrackexamplechirpparams} for the chirp parameters.
    \label{plot:mq_lp_compare_chirp_10}}
\end{figure}

A spectrogram of each signal is computed with an analysis window length of 1024
samples and a hop-size $H$ of 256 samples. Local maxima are searched in 150 Hz
wide bands spaced 75 Hz apart. A local maximum is only accepted if its amplitude
is greater than -20 dB. The bin corresponding to each local maximum and its two
surrounding bins are used by the Distribution Derivative Method
(DDM) to estimate the
local chirp parameters, the $i$th set of parameters in frame $k$ denoted
$\theta_{i}^{k} = \left\{ \phi_{i}^{k} , \omega_{i}^{k} , \psi_{i}^{k}
\right\}$ (the atoms used by the DDM are generated from 4-term continuous
Nuttall windows). The results of the analyses of both signals are lumped together and
it is on this lumped data that we perform partial tracking.

We search for partial tracks using both the greedy and LP strategies. Both
algorithms use the distance metric $\mathcal{D}_{\text{pr.}}$ between two parameters sets:
\[
    \mathcal{D}_{\text{pr.}} \left( \theta_{i}^{k},
    \theta_{j}^{k+1} \right) = \left( \omega_{i}^{k} +
    \psi_{i}^{k} H - \omega_{j}^{k+1} \right)
\]
which is the error in predicting $j$th frequency in frame $k+1$ from the $i$th
parameters in frame $k$. For the greedy method, the search for partial paths is
restricted to one frame ahead like in \cite{mcaulay1986speech}, otherwise the
computation becomes intractable. For the LP
method, to keep the computation time reasonable, we search over 6 frames for 6
best paths (the number of paths does not affect the computation time).
To maintain connected paths, the search on the next frames uses the end nodes of
the last search as starting points. For both methods, the search is restricted
to nodes between frequencies 250 to 2250 Hz.

Figure~\ref{plot:mq_lp_compare_chirp_10}
shows discovered partial trajectories for signals with a SNR of 10
dB. It is seen that while the greedy method performs poorly
at a SNR of 10dB, the LP method still gives plausible partial trajectories.

\section{Conclusion}

In this paper we reformulated the classical greedy algorithm of McAulay and
Quatieri and showed that it can be seen as a greedy algorithm for finding the $L$
shortest paths in a lattice. An algorithm was then proposed minimizing the sum
of the $L$ paths, using a linear programming approach. It was shown on synthetic
signals that the new approach finds plausible paths in lattices with a
large number of spurious nodes.

There are problems with the proposed approach. Perhaps ``jagged'' paths should be removed using regularization.
There are also situations where it is undesirable to have paths extend
throughout the entire lattice. Acoustic signals produced by striking media, such
as strings or bars, exhibit a spectrum where the upper partials decay more
quickly than the lower ones --- it
would be desirable in these situations to have shorter paths for the upper
partials, those decaying more quickly. This could be addressed as in
\cite{depalle1993tracking} where the signal is divided into overlapping sequences of
frames and partial paths are connected between sequences.

The proposed algorithm, while faster than algorithms based on the Viterbi
algorithm, is still not fast. In many situations where computational resources are
limited, a McAulay-Quatieri method search over many sets of small $K$ works
sufficiently well. However in high amounts of noise the algorithm proposed here
is robust and of tractable complexity.

%\section{ACKNOWLEDGMENT}

% -------------------------------------------------------------------------
% Either list references using the bibliography style file IEEEtran.bst
\bibliographystyle{IEEEtran}
\bibliography{paper}

\end{sloppy}
\end{document}
